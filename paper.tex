\documentclass{mcmthesis}%美赛模板
\mcmsetup{CTeX = false,   % 使用 CTeX 套装时,设置为 true
	tcn = 2023-2, problem = A,%队伍控制号码,接受一个字符串作为值;选题,接受一个字符串作为值;
	sheet = true, %为真时将输出摘要页,否则不输出;默认为 true。
	%color = red,  %设置控制页的题目号的颜色
	titleinsheet = true, %为真时将在摘要页输出标题,否则不输出;默认为 false。
	keywordsinsheet = true,%为真时将在摘要页输出关键字,否则不输出;默认为 false。
	titlepage=false,%为真时将输出标题页,否则不输出;默认为 true。
	abstract=true}%为真时将在标题页输出摘要和关键词,否则不输出;默认值为 true。
\usepackage[T1]{fontenc}%fontenc 宏包是用来配合传统的LaTeX字体的,如上表中的一些传统字体宏包。如果使用xelatex编译方式,并使用fontenc宏包调用ttf或otf格式字体,就不要再使用fontenc宏包. 
\usepackage{palatino}  %控制正文字体,若是不喜欢可以注释掉。
\usepackage{lipsum} %输入中文乱数假文。乱数假文就是大段无意义的文字,常用来测试排版效果。
\usepackage{amsmath}%定义数学公式的宏包
\usepackage{amssymb}%也是定义数学公式的一个宏包,对amsmath进行补充
\usepackage{subfig}%排列图片的宏包
\usepackage{float}%避免浮动体宏包
\usepackage{indentfirst} %设置首行自动缩进
\newcommand{\itemEq}[1]{%
	\begingroup%
	\setlength{\abovedisplayskip}{0pt}%
	\setlength{\belowdisplayskip}{0pt}%
	\parbox[c]{\linewidth}{\begin{flalign}#1&&\end{flalign}}%
	\endgroup}%itemiz里面套equation
\setlength{\parindent}{2em}
\title{Type Your Paper Title HERE  }  %论文的题目
\author{A \and B \and C}
\date{\today}

\makeatletter
\renewcommand*\l@section{\@dottedtocline{1}{12pt}{12pt}}
\newcommand{\upcite}[1]{\textsuperscript{\textsuperscript{\cite{#1}}}}
\makeatother
\begin{document}    %文档的开头
	\begin{abstract} %此处默认全部使用英文,若要使用中文显示,需要换成:\documentclass{ctexart}
		\small
        %%你的摘要内容
	\begin{keywords}   %关键字
			\footnotesize %设置字体
		%%关键字
	\end{keywords}
	\end{abstract}
	
	\maketitle  %显示正文中出现了标题和作者
	\tableofcontents %显示目录
	\clearpage  %在latex中遇到很长的英文单词,仅在单词之间的”空格“初断行无法生成疏密程度均匀段落时,就会考虑从单词中间断开。对于绝大多数单词,latex能够找到合适的断词位置,在断开的行尾加上连字符-。
	\normalsize  %重新定义默认字体大小
	
	%1.引言
	\section{Introduction}
    \subsection{ Question Restatement}
    \subsubsection{title}
    \subsection{Analysis of the Question}
    
    %2.假设和理由
	\section{ Assumptions and Justifications}
	
	%3.模型1
	\section{ Model 1}
	\subsection{ Subtitle}
	%4.模型2
	\section{Model 2}
	%5.模型x
	\section{Model x}
	
	%6.优点和缺点
	\section{Strengths and Weaknesses}
	
	%7.结论
	\subsection{Conclusion}

    %8.策略建议信
	\section{Strategy Recommendation and Conclusions}
	\noindent Theme: %%题目
	\begin{letter}{Dear Director,}
        
        %%建议信的正文
		
		\vspace{\parskip}
		Sicerely yours,\\
		Your friends
	\end{letter}
	
	%9.文献
	\begin{thebibliography}{99}
		\bibitem{1}
		\bibitem{2}
		\bibitem{3}
		\bibitem{4}
	\end{thebibliography}
	
	%附录,但美赛规定不添加附录,所以以下可以忽略
	\begin{appendices}
		\section{Code for Part 3.1}
		\begin{lstlisting}[language=python]
			import pandas as pd
			import numpy as np
			import matplotlib.pyplot as plt
		\end{lstlisting}
		
		\section{Code for Part 3.3}
		\begin{lstlisting}[language=python]
			import pandas as pd
			import numpy as np
			import matplotlib.pyplot as plt	
		\end{lstlisting}	
	\end{appendices}


\end{document}
